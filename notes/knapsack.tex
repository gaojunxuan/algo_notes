\section{Knapsack} \index{knapsack problem}

In the knapsack problem, we are given a set of $n$ items $I_1,\ldots,I_n$ and a size bound $B$ where each item $I_j=(w_j,v_j)$ with $w_j$ being the weight of the item and $v_j$ the value of the item.

A subset of items $S$ is feasible if the sum of the weights of items in $S$ is at most $B$. The goal of the knapsack problem is to find a feasible set $S$ that maximizes the sum of the values of items in $S$.

\section{DP Algorithm for 0-1 Knapsack}

We define the semantic array
$$
V(i,b) = \text{max profit possible using the first $i$ items within weight $b$}
$$
with corresponding Bellman equation
$$
V(i,b) = \begin{cases}
    0 & \text{if $i=0$ or $b=0$} \\
    \max\{C,D\} & \text{if $w_i \leq b$}
\end{cases}
$$
where $C = V(i-1,b)$ and $D = V(i-1,b-w_i) + v_i$.

$C$ corresponds to the decision to not include the $i$th item in the knapsack, and $D$ correspond to the decision to include the $i$th item, in which case we deduce $w_i$ from the remaining weight of the knapsack and add $v_i$ to the total profit. Like all DP problems, we need to prove that the semantic array is equivalent to the Bellman equation.

\begin{codebox}
    \Procname{$\proc{Knapsack}(S,B)$}
    \li $n = \attrib{S}{length}$ 
    \li \For $b=0$ to $B$ \Do
        \li $M[0,b] = 0$
    \End
    \li \For $i=1$ to $n$ \Do
        \li \For $b = 0$ to $B$ \Do
            \li \If $S[i].w > b$ \Then
                \li $M[i,b] = M[i-1,b]$
            \li \Else
                \li $M[i,b] = \max\{M[i-1,b],\, S[i].v + M[i-1,b-1]\}$
            \End
        \End
    \End
    \li $M[n,B]$ 
\end{codebox}

The input to the algorithm: $S$ is the set of all items to be considered, and $B$ is the weight bound of the knapsack. This approach makes an important assumption that the weights are integer.

\index{pseudopolynomial}
This DP algorithm solving the 0-1 knapsack problem with $n$ items and weight bound $B$ runs in $\Theta(nB)$ time and uses $\Theta(nB)$ space. As we can see, the running of this algorithm is, unfortunately, not polynomial in the input size. It is \textbf{pseudo-polynomial}. It is polynomial in $\log B + \sum_{i=1}^n (\log v_i + \log w_i)$.

If $B$ is small (a polynomial in $n$), then the algorithm runs in polynomial time. No known exact algorithm runs in polynomial time. The 0-1 knapsack problem is NP-complete.

\section{A Different DP Algorithm}

Consider version of the knapsack problem with a different restriction that the values $v_i$ are integer and small ($V=\sum_{i=1}^n v_i \ll B$). In this case, it makes more sense to derive an algorithm in terms of $v$ instead of $B$.

The semantic array is defined as follows
$$
W(i,v) = \begin{cases}
    \substack{\text{minimum weight required to obtain at least} \\ \text{profit $v$ using a subset of $\{I_1,\ldots,I_i\}$}} & \text{if possible} \\
    \infty & \text{otherwise}
\end{cases}
$$
The goal is to compute $\max\{v \mid W(n,v) \leq B\}$.

The corresponding Bellman equation is
$$
W(i,v) = \begin{cases}
    \infty & \text{if $i=0$ and $v>0$} \\
    0 & \text{if $i\leq 0$ or $b\leq 0$} \\
    \max\{C,D\} & \text{otherwise}
\end{cases}
$$
where $C = W(i-1,v)$ and $D=W(i-1,v-v_i) + w_i$.

This algorithm is still pseudo-polynomial but the complexity is now $O(nV)$ where $V = v_1 + \cdots + v_n$. This is more efficient when $V \ll B$.

\section{FPTAS Approximation for Knapsack}

Because 0-1 knapsack is NP-complete, it is also of theoretical interest to find an efficient approximation algorithm. In the case of the knapsack problem, it is possible to find an algorithm that approximates the result within a specific degree on all possible inputs.

The idea behind the algorithm is as follows: The high order bits/digits of the value $v$ can be used to determine an approximate solution. The fewer high order bits we use, the faster the algorithm runs, but also the worse the approximation.

The goal is to scale the values $v$ using a parameter (rounding factor) $\epsilon$ so that a $(1+\epsilon)$ approximation is obtained with time complexity polynomial in $n$ and $1/\epsilon$.

More precisely, let $\tilde{v} = \lceil v_i/b \rceil b$. This ensures that the constraint that $v$ must be integer for the second algorithm is satisfied. By rounding the values, we get an approximation $\tilde{v}$ of $v$ that is divisible by $b$. This means we can divide all values of $v$ by $\epsilon$ and get an equivalent problem. Let $\hat{v} = \tilde{v}/b$. This allows us to reduce the size of $v$ so that it becomes smaller than $n$ (or at least bounded by a polynomial in $n$).

\begin{codebox}
    \Procname{$\proc{Knapsack-Approx}(\epsilon,S)$}
    \li $b = (\frac{\epsilon}{2n}) \cdot \max_i v_i$
    \li \For $i$ in $S$ \Do 
        \li $\tilde{v} = \lceil v_i/b \rceil b$
        \li $\hat{v} = \tilde{v}/b$
        \li $i.v = \hat{v}$
    \End
    \li $\proc{Knapsack-V}(S,\max_i\{v_i\})$ 
\end{codebox}
where $\proc{Knapsack-V}$ is the second DP algorithm discussed above.

\proc{Knapsack-Approx} is a \textbf{FPTAS} (fully polynomial time approximation scheme) algorithm for the 0-1 knapsack problem.

\begin{definition}[FPTAS and PTAS] \index{FPTAS} \index{PTAS}
    \hfill \\
    An \textbf{FPTAS} (Fully Polynomial Time Approximation Scheme) algorithm is one that is polynomial in the encoding of the input and $1/\epsilon$.

    A \textbf{PTAS} (Polynomial Time Approximation Scheme) algorithm is one that is polynomial to the encoding of the algorithm but can have any complexity in terms of $1/\epsilon$ ($1/\epsilon$ term may not be polynomial).
\end{definition}